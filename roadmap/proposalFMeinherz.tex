\documentclass[11pt, UKenglish]{report}
\usepackage[a4paper]{geometry}
\usepackage{lmodern}
\renewcommand*\familydefault{\sfdefault}
\usepackage[T1]{fontenc}
\usepackage[UKenglish]{babel}
\usepackage[utf8]{inputenc}
\usepackage{hyperref}
\usepackage{pdfcomment}

\newcommand{\HRule}[1]{\rule{\linewidth}{#1}}

\begin{document}

%-------------------------------------------------------------------------------
% TITLE
%-------------------------------------------------------------------------------

\noindent \large Math-710 Data analysis for science and engineering

\noindent Spring 2017

\vspace{0.5cm}
\noindent \LARGE \textbf{Proposal for the final project}

\normalsize
\vspace{1cm}
\noindent Submitted by \large{Franziska Meinherz}

\vspace{0.5cm}


%-------------------------------------------------------------------------------
% PROPOSAL
%-------------------------------------------------------------------------------

% PRÉAMBULE
%-------------------------------------------------------------------------------

\noindent \HRule{0.5pt} \\
\large{\textit{\textbf{Preliminary note:}}

\normalsize

\noindent \textit{Due to a slight reorientation in my PhD (I am still in the 1\textsuperscript{st} year), I am now at a point where I need to do a lot of conceptual work. I do not currently have research questions requiring statistical analysis. Those will emerge out of the conceptual work in which I am currently engaged. Therefore, for this semester project, I would like to work on a very pressing problem from an association of which I am part (\href{http://octanis.org/rover/} {Octanis Association - Rover project}). I am confident that applying the methods and concepts which we have learnt in class to this problem will enhance my practical understanding of them such that I can then easily apply them to my PhD, once I am more advanced in it.}

\noindent \HRule{0.5pt}

% PROBLEM
%-------------------------------------------------------------------------------

\section*{Research problem}

\subsection*{Problem context}

Octanis Association has built a test rover which can perform data gathering tasks in difficult terrain, and which can be built with basic and low-cost components and runs with open source software. At the moment, the test rover has the sensors required for gathering data on the structure of the snow cover of any given area, and has been sent on a mission  to Antarctica undertaken by researchers from the University of Grenoble. The rover is powered by solar cells which are mounted on its top and has a battery to store energy. The data which it gathers is stored on an SD card which is built into the rover. The rover both gathers data regarding its task - in this case, the structure of the snow cover - and on its own performance.

The rover will be back from its current mission towards the end of May. As a follow-up to this mission, the current rover will be improved, in order to enhance its usefulness for similar missions. One key challenge is to enhance the time during which the rover can be active, by optimising the process of charging and discharging the battery.

\subsection*{Problem statement}

The aim of this research project is to use the data which the rover has gathered on its battery and battery charging / discharging status, as well as on its context, to develop an optimising procedure for when the rover should rest and charge its batteries, and for when it should be active and use energy, in order to maximise its operation time. The project thus needs to address the following points:

\begin{enumerate}

	\item{\large{Data extraction and structuring:} \normalsize The data collected by the rover is stored in ROS .bag files, is ordered by different topics, and can be accessed through command line tools which are comprised in the ROS package (see additional details under "Data"). The data thus has to be extracted into .csv files and organised and structured in a way that facilitates its manipulation with R and which corresponds to the purposes of this semester project.}

	\item{\large{Data exploration:} \normalsize Since the current rover is a pilot project, it seems appropriate to develop tools to visualise the data in ways which allow to identify potential measuring or storing errors, and which additionally provide an overall picture of the rover's functioning, in order to develop first hypotheses as to how its operation time could be enhanced.}

	\item{\large{Data analysis:} \normalsize Data on the rover's battery charge status and battery charging / discharging has to be compared to data on the current UV index /sam{luminosity and solar panel voltage (which varies in function of how bright it is, angle of light rays) more appropriated, not uv index}  and other indicators for battery charging potential, on wind intensity and direction\pdfcomment[icon=Note]{wind data is not collected, but could be downloaded from online weather platforms if really desired} in comparison to the rover's direction of movement, outside temperature, and the rover's speed, in order to find to an optimising strategy for the conditions under which the rover should rest and charge its battery, and for when it should move and use energy.}

\end{enumerate}

% DATA
%-------------------------------------------------------------------------------

\section*{Data}

Since the arrival time of the rover and the data it has gathered during its mission is scheduled only for the end of May, for this project, data from test runs undertaken in Renens will be used. The data structure is exactly the same between the data gathered during the test runs in Renens and the actual mission in Antarctica. Therefore, it should not matter which is used as an input to the data analysis script which is to be developed in the course of this project, with the exeption that no pertinent conclusions can be drawn from the test runs, since the setting was artificial and the sample size - in this case the amount of time which the rover spent in activity - is too small /sam: {not necessarily true. the amount of time spent collecting data has less impact than the terrain the data is collected on. we will not have significantly more hours of data from the real mission, so no emphasis on this should be given here}.

\subsection*{Data format and structure}

The data is stored in .bag files. This file structure has been developed as part of the ROS (Robot Operating System) environment to store data collected by any robot as soon as it is received. The data is thus organised along a timeline which corresponds to the robot's internal clock (although this can be overridden with a simulated clock)\pdfcomment[icon=Note]{has this been done? /sam: {data is timestamped arbitrarily and it doesn't matter to your analysis if there is an offset}}. In this process, the data is organised in different topics, which each can store different variables. The data can be accessed through different command line tools which are available in the ROS package and can be played back as text in the command window. The data is accessed by topic. The ROS package also provides some tools to visualise the played back data.

\subsection*{Variables}\pdfcomment[icon=Note]{not even sure this is needed at this stage but is good for us to know anyways so we can clarify it even if in the end I don't include it in the proposal}

For this project, two kinds of variables will be considered: On the one hand, variables which relate to the rover's functioning, such as its battery status, its battery charging / discharging status, its speed, its energy use per achieved motion, and its direction, and on the other hand, variables which describe the rover's surroundings and the situational context, such as variables describing the available solar power, variables on wind intensity and direction\pdfcomment[icon=Note]{same as above - do we have this data? - sam: no wind}, information on the structure of the terrain - both on the macro level (elevation and inclination) and the micro level (snow cover structure), and the current air pressure.\pdfcomment[icon=Note]{does this list remotely correspond to what is available? btw I seem to remember that you showed me a list of the variables somehow - can you print screen this thing or send it or whatever? }

/sam:  data available is. i would leave out the laser scanner data (snow cover structure data) as this is a very advanced form of data to look at and is a project on its own. extract of "easy" available data is below:

- ENERGY
	- BATTERY_STATUS
		- Current in (current going into battery from solar -> charging)
		- Current out (current going into rover system)
		- battery_remaining in percent (voltage_battery)
		- solar panel voltage (voltage measured at the solar panels (how much "sun" do we have right now?))
		- est. solar power
-WEATHER
		 - air pressure
		 - temperature
		 - humidity
		- UV INDEX
		- Lux
- POSE
	- latitude, longitude
	- acceleration x y z
	- magnetometer x y z
	- gyroscope x y z

dashboard here:<
http://octanis.org/constellation/




% METHODS
%-------------------------------------------------------------------------------

\section*{Methods}

\subsection*{Data extraction and formatting}

\begin{enumerate}

	\item{\large{Data extraction:} \normalsize böh? \pdfcomment[icon=Note]{teach me!!}}

	\item{\large{Data structuring:} \normalsize  The .csv files obtained in the previous step are to be organised in such a way that the data is accessible in ways which makes sense given the steps which are to be performed in the data analysis part. As the project advances, the decisions regarding the formatting of the data are thus revisited in an iterative process.}

\end{enumerate}

\subsection*{Data exploration}

\begin{enumerate}

	\item{\large{Finding measuring / storing errors:} \normalsize Data must be checked for invalid entry points as well as for unreasonable outliers. For the first, böh?!?! \pdfcomment[icon=Note]{help - does the ROS package provide something to find invalid entries or would it have to be done after conversion to csv?}. For the second, the data is to be visualised in box plots. In cases where realistic data ranges cannot easily be determined, the distribution as shown in the box plots is used as an indicator for what can be assumed a realistic data range, and outliers to this derived realistic range are then scrutinised for their likelihood to be erronic values. Depending on the variable, this can be done by looking at the values recorded shortly before and after, or by cross-comparing the outliers by looking at their relative similarity as well as at the values of the other variables recorded at the same time amongst similar outliers. In cases where realistic data ranges can easily be determined, outliers lying outside of them are to be eliminated.}

	\item{\large{Providing an overall picture:} \normalsize The aim of this step is to draw a picture of the rover's functioning depending on the situational context. Therefore, the variables describing the rover's battery status, and battery charging / discharging are to be plotted against variables describing the situational context (weather data etc.) as well as against variables describing the rover's activity (speed, direction, elevation gain / loss etc.)}

\end{enumerate}

\subsection*{Data analysis}

Different analytical tools are to be used in order to gain a picture which is as consistent as possible with what is actually influencing the rover's battery lifetime.

\begin{enumerate}

	\item{\large{Correlation analysis:} \normalsize Variables describing the rover's functioning are constrasted with variables describing the situational context in a cross-correlation analysis, in order to identify on the one hand whether there seem to be patterns in how they co-develop, and on the other hand, to identify the situational variables which most seem to influence the rover's functioning.}

	\item{\large{Factor analysis \pdfcomment[icon=Note]{applying the classic method from sociology to robotics <3}/ Principal component analysis:} \normalsize The aim of this exercise is to see whether there are clusters emerging describing specific modes of functioning in specific circumstances. Such clusters could in a next step be used to identify optimal strategies for the rover. This exercise is to be done both through a principal component analysis, where the emerging clusters are then analysed regarding their logical meaning and consistency, and through a factor analysis, where the factors are predefined based on the insights from the correlation analysis and hypotheses regarding the logic underlying the rover's functioning, and then tested regarding their explanatory power, internal consistency, and significance.}

\end{enumerate}

\pdfcomment[icon=Note]{the aim of the data analysis part is to provide the input for some kind of optimising procedure. but I think this goes beyond the scope of the data analysis class, but could be useful for the rover improvement, as far as I understood?}

\end{document}
